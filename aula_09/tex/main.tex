\UseRawInputEncoding
\documentclass[12pt,a4paper]{article}
\RequirePackage[T1]{fontenc}
\RequirePackage[english]{babel}
\RequirePackage[utf8]{inputenc}
\usepackage{leading}
\usepackage{multicol} 
\usepackage{booktabs, cellspace, multirow}
\usepackage{indentfirst}
\usepackage[pdftex]{hyperref}
\usepackage{ae}
\usepackage[pdftex]{graphicx}
\usepackage[pdftex]{geometry}
\usepackage{cite}

\usepackage{hyperref}
\hypersetup{
	colorlinks=true,
	linkcolor=blue,
	filecolor=magenta,      
	urlcolor=black,
	citecolor=black,
	pdfpagemode=FullScreen,
}

\fontfamily{pcr}

\renewcommand*{\arraystretch}{1.1}

\setlength\arrayrulewidth{0.7pt}
\setlength{\parindent}{0.9cm}
\setlength\aboverulesep{2pt}
\setlength\belowrulesep{2pt}
\setlength\cellspacetoplimit{3pt}
\setlength\cellspacebottomlimit{3pt}
\geometry{a4paper,left=1in,right=1in,top=1cm,bottom=2cm}

\title{\textbf{Lista de Exercícios - Desafio sem Resposta}\\Em grupo de até cinco alunos}
\author{Prof. MSc. Edson Melo de Souza}
\date{\vspace{-5ex}}

\begin{document}
	\bibliographystyle{apalike} 
	\maketitle
	\thispagestyle{empty}

	\section{Lógica Proposicional}
		1) Analise a tabela abaixo e indique qual sentença corresponde a coluna ``Sentença''.
		\begin{table}[h!]
		\centering

		\begin{tabular}{|c|c|c|c|}
			\hline
			p & q & r & Sentença \\ \hline
			1 & 1 & 1 & 1        \\ \hline
			1 & 1 & 0 & 1        \\ \hline
			1 & 0 & 1 & 0        \\ \hline
			0 & 1 & 1 & 1        \\ \hline
			0 & 0 & 1 & 1        \\ \hline
			0 & 1 & 0 & 1        \\ \hline
			1 & 0 & 0 & 0        \\ \hline
			0 & 0 & 0 & 1        \\ \hline
			\end{tabular}%
		
		\end{table}
		\begin{enumerate}
			\item \( ( p \to r ) \wedge ( q \to r ) \neg q \leftrightarrow q \)
			\item \( ( r \leftrightarrow p ) \vee ( r \to q ) \neg p \leftrightarrow r \)
			\item \( ( p \to q ) \vee ( q \leftrightarrow r ) \neg q \to r \) %essa
			\item \( ( p \neg r ) \vee ( q \vee r ) \vee q \leftrightarrow p \)
			\item \( ( r \neg \neg q ) \neg ( p \vee q ) \vee r \leftrightarrow p \)
		\end{enumerate}

		2) Utilizando o Python, desenvolva um programa que receba dois valores e monte a tabela verdade para cada um dos casos. Utilize ``listas'', ``repetições'' e ``operadores'' para resolver:
		\begin{enumerate}
			\item \( p \neg q \)
			\item \( p \leftrightarrow q \)
			\item \( (p \to q) \wedge \neg q\)

		\end{enumerate}

	\section{Implementação}
		Utilizando a linguagem Python, resolva os exercícios propostos a seguir.

		\textbf{Observação}: Utilize a ferramenta \textit{online} \href{https://repl.it/}{repl.it (https://repl.it)}
	
		\begin{enumerate}
			\item (\textbf{Very Easy}) Faça um Programa que peça dois números e imprima o maior deles.
			\item (\textbf{Easy})Faça um Programa que peça um valor e mostre na tela se o valor é positivo ou negativo.
			\item (\textbf{Easy})Faça um Programa que verifique se uma letra digitada é vogal ou consoante.
			\item (\textbf{Easy}) Faça um Programa que leia três números e mostre o maior deles.
			\item (\textbf{Medium}) Faça um Programa que leia três números e mostre o maior e o menor deles.
			\item (\textbf{Medium}) Faça um programa que pergunte o preço de três produtos e informe qual produto você deve comprar, sabendo que a decisão é sempre pelo mais barato.
			\item (\textbf{Easy}) Faça um Programa que leia três números e mostre-os em ordem decrescente.
			\item (\textbf{Easy}) Faça um Programa que pergunte em que turno você estuda. Peça para digitar M-matutino ou V-Vespertino ou N-Noturno. Imprima a mensagem "Bom Dia!", "Boa Tarde!" ou "Boa Noite!" ou "Valor Inválido!", conforme o caso.
			\item (\textbf{Hard Level}) Uma pessoa foi ao supermercado e comprou diversos produtos. Como era estudante de Ciências da Computação, desenvolveu um programa que fizesse os cálculos automaticamente dos produtos comprados. A seguir, é mostrada a lista dos produtos comprados. Usando o conceito de variáveis e outros aprendidos até o momento, mostre os valores totais por produto, acumulado e o total gasto na compra.

\begin{table}[h!]
\centering
\begin{tabular}{|l|l|l|l|l|l|}
\hline
\textbf{Produto} & \textbf{Valor} & \textbf{Quantidade}                                                           & \textbf{Desconto} & \textbf{Total} & \textbf{Total Acumulado} \\ \hline
Azeitona         & R\$ 9,45       & 3                                                                             &                   &                &                          \\ \hline
Leite            & R\$ 3,34       & \begin{tabular}[c]{@{}l@{}}12\\    \\ 5\% \textgreater{}= 4 uni.\end{tabular} &                   &                &                          \\ \hline
Açúcar           & R\$ 2,12       & \begin{tabular}[c]{@{}l@{}}10\\    \\ 3\% \textgreater 10 uni.\end{tabular}   &                   &                &                          \\ \hline
Farinha          & R\$ 4,67       & 2                                                                             &                   &                &                          \\ \hline
Fermento         & R\$ 1,88       & 4                                                                             &                   &                &                          \\ \hline
Vinagre          & R\$ 5,56       & \begin{tabular}[c]{@{}l@{}}2\\    \\ 10\% \textgreater{}= 2\end{tabular}      &                   &                &                          \\ \hline
\multicolumn{5}{|l|}{Total Geral}                                                                                                                      &                          \\ \hline
\end{tabular}
\end{table}
 		\end{enumerate}		
\end{document}